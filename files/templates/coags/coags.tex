\documentclass{standalone}
\usepackage{tikz}
\usetikzlibrary{calc, positioning}

\begin{document}
	
	\begin{tikzpicture}[node distance=0mm]
		\def\blank{Y};
		\ifx\blank\empty
			\tikzstyle{data} = [black]
			\def\pt{PT};
			\def\inr{INR};
			\def\ptt{PTT};
		\else
			\tikzstyle{data} = [white]
			\def\pt{11.5};
			\def\inr{35.2};
			\def\ptt{0.9};
		\fi

		\node (center) at (0,0) {};

		\node[data, left of=center, anchor=south east] (pt) {\pt};
		\node[data, right of=center, anchor=south west] (ptt) {\ptt};
		\node[data, below of=center, anchor=north, yshift=-2] (inr) {\inr};

		\draw
			let 
				\p1 = (center),
				\p2 = (pt.north),
				\p3 = (ptt.north),
				\p4 = (inr.south)
			in
				(\x1,\y1+2) -- (\x1,{max(\y2,\y3)+4});
		\draw
			let 
				\p1 = (center),
				\p2 = (pt.north),
				\p3 = (ptt.north),
				\p4 = (inr.south west)
			in
				(\x1,\y1+2) -- ({\x1+1.25*\x4},2/3*\y4);
		\draw
			let 
				\p1 = (center),
				\p2 = (pt.north),
				\p3 = (ptt.north),
				\p4 = (inr.south west)
			in
				(\x1,\y1+2) -- ({\x1-1.25*\x4},2/3*\y4);
	\end{tikzpicture}

\end{document}